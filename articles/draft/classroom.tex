\documentclass[12pt]{article}

\usepackage[utf8]{inputenc}
\usepackage{hyperref}

\author{Salvador Guzm\'{a}n}
\date{\today}
\title{Schemata for Effective Classroom Governance}

\begin{document}
  \maketitle
  \pagebreak
  \tableofcontents
  \pagebreak

  \section{Introduction}
  \section{Purpose}

  \section{Who Owns the Classroom?}
  \subsection{Introduction}
  \subsection{Thesis}
  \subsection{The Current Classroom}
  \subsection{The Restorative Approach}
  \subsection{Questions}
  \subsection{Attributes of a Successful Classroom}
  \subsection{Speculative Mitigation Strategies}
  \subsection{Social Dynamics of Students}

  \section{Introduction}
  \section{Thesis}

  \section{Current Classroom Deficiencies}
  \subsection{A Priori Institutional...}
  \subsubsection{Insufficient institutional Resources to control discipline-challenged Students}
  \subsubsection{Disconnect Between Student-Centered Goals and Corporate Goals}
  \subsubsection{Opaque Management Decision Process}
  \subsubsection{Priority of Academic Performance at the Expense of Discipline}
  \subsubsection{Resisting the Existence of Adversarial Dynamics Between Students and Staff}
  \subsubsection{Institutional a Priori Deference to Students}
  \subsubsection{Teachers' time is seen as an Elastic Resource}
  \subsubsection{Application of Mitigation Strategies that Assume Student Cooperation}

  \subsection{Post-Hoc Classroom Observable Shortcomings}
  \subsubsection{Trouble Students Controlling Class Climate}
  \subsubsection{Transfer of Ownership of Classroom Dynamics to Students}
  \subsubsection{Endemic Student non-cooperation}
  \subsubsection{Student Insubordination}

  \section{The Restorative Approach, Dissected}
  \subsection{Chapter 1}
  \subsection{Chapter 2}
  \subsection{Chapter 3}
  \subsection{Chapter 4}
  \subsection{Chapter 5}
  \subsection{Chapter 6}

  \section{Questions that require an answer}
  \subsection{Pedagogical Questions}
  \subsubsection{What are the alternatives to this approach?}
  \subsubsection{What is the criteria for selection?}
  \subsubsection{Is pedagogical counter-narratives allowed or addressed within the scope of the book?}
  \subsubsection{Why was this specific approach chosen?}
  \subsubsection{What would it take to falsify the efficacy of this approach?}
  \subsection{Efficacy Related Questions}
  \subsubsection{Has the restorative approach demonstrated success?}
  \subsubsection{Has the restorative approach demonstrated failure?}
  \subsubsection{What is the track record of this approach?}
  \subsubsection{Does the restorative approach address classroom governance?}
  \subsection{Tonal Questions}
  \subsubsection{What was the intent of producing this piece of literature on this approach?}
  \subsubsection{Who is the author?}
  \subsubsection{Does the author have a stake in the implementation of this approach?}
  \subsubsection{How does the author benefit from use of this approach?}
  \subsubsection{Does the tone allow any space for dissent?}
  \subsubsection{Does the tone address any criticism?}
  \subsubsection{Is intellectual criticism possible within the scope of the literature?}

  \section{Attributes of a Successful Classroom}
  \subsection{Positive Characteristics}
  \subsubsection{Teacher Morale}
  \subsubsection{Student Morale}
  \subsubsection{Classroom Ownership}
  \subsubsection{Classroom Social Dynamics}
  \subsubsection{Disciplinary Pipeline to Centralize Troublesome Students' Outbursts}
  \subsection{Negative Characteristics}
  \subsubsection{Does not Offload Bulk Disciplinary Effort onto Teachers}
  \subsubsection{Does not Designate Teachers as Head's of Discipline}
  \subsubsection{Does not Force Teachers to Innovate to Achieve Disciplinary Ends}
  \subsubsection{Does not See Teachers' Time as an Elastic Resource to be Tapped into}

  \section{Attributes of a Failed Classroom}
  \subsection{Formation of Political Associations Among Students Competing with Instructor for Authority}
  \subsection{Injection of Sexually Charged Comments}

  \section{Pushback}

  \section{Speculative Mitigation Strategies}

  \section{Social Dynamics of Students}
  \subsection{Hypersociality of Our Students}
  \subsection{Hypersociality of Our Students}
  \subsection{Tracing the Contours of Power Within Student Associative Groups}
  \subsection{Opportunistic Insubordination In the Presence of Student Authority Figures}

  \section{Conclusion}
  \pagebreak

\end{document}